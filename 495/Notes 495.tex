\documentclass[12pt,twoside]{article}

\newcommand{\reporttitle}{495 Advanced Statistical Machine Learning and Pattern Recognition}
\newcommand{\reportauthor}{Thomas Teh}
\newcommand{\reporttype}{Notes}
\newcommand{\cid}{0124 3008}

% include files that load packages and define macros
\input{includes} % various packages needed for maths etc.
\input{notation} % short-hand notation and macros


%%%%%%%%%%%%%%%%%%%%%%%%%%%%

\begin{document}
% front page
\input{titlepage}


%%%%%%%%%%%%%%%%%%%%%%%%%%%% Main document
\section{Expectation Maximization}

\subsection{General Approach for Expectation Maximization}
\subsubsection{Notes}
\begin{enumerate}
	\item The goal of the Expectation-Maximization is to find maximum likelihood solutions for models having latent variables
	\item The general concept is that since our knowledge of the latent variables in $\vec{Z}$ is given by the posterior distribution $p(\vec{Z}\vert \vec{Z}, \vec{\theta})$, we use the expectation of the latent variables instead of the actual values.
	\item EM algorithm can be used to find MAP solutions models in which a prior is defined over the parameters.
\end{enumerate}

\subsubsection{Algorithm}
Given a joint distribution $p(\vec{X}, \vec{Z}\vert \vec{\theta})$ over observed variables $\vec{X}$ and latent variables $\vec{Z}$, governed by parameter $\vec{\theta}$ the goal is to maximize the likelihood function $p(\vec{X}\vert \vec{\theta})$ w.r.t $\vec{\theta}$.
	\begin{enumerate}
		\item Choose an initial for the parameters $\vec{\theta}^{old}$.
		\item E Step: Evaluate $p(\vec{Z} \vert \vec{X}, \vec{\theta}^{old})$
		\item M Step: Evaluate $\vec{\theta}^{new}$ given by
			\begin{align*}
				\vec{\theta}^{new}& = \text{arg} \max_{\vec{\theta}} \mathcal{Q}(\vec{\theta},\vec{\theta}^{old})
			\end{align*}
		where
			\begin{align*}
				\mathcal{Q}(\vec{\theta},\vec{\theta}^{old})& = \sum_{\vec{z}}p(\vec{Z}\vert \vec{X}, \vec{\theta}^{old})\ln p(\vec{X}, \vec{Z} \vert \vec{\theta})
			\end{align*}
		\item Check for convergence of either the log-likelihood or the parameter values. If convergence is not satisfied
				\begin{align*}
					\vec{\theta}^{old} \leftarrow \vec{\theta}^{new}
				\end{align*}
		and return to step 2
\end{enumerate}


\subsection{Gaussian Mixture Models}
The Gaussian mixture distribution can be written as a linear superposition of Gaussians
\begin{align*}
	p(\vec{x}) &= \sum_{k=1}^K \pi_k \mathcal{N}(\vec{x}\vert \vec{\mu}_k, \vec{\Sigma}_k)
\end{align*}

\begin{figure}[H]
	\begin{center}
		\includegraphics[width = 0.4\hsize]{./figures/GMM.png} 
		\caption{Gaussian Mixture Model.} % caption of the figure
		\label{fig:GMM} % a label. When we refer to this label from the text, the figure number is included automatically
	\end{center}
\end{figure}


\subsubsection{Formulation of the Gaussian Mixture Models}
Let $\vec{z}$ be a $K$-dimensional binary random variable with 1-of-$K$ representation.
\begin{align*}
	p(z_k=1) = \pi_k 	&\Rightarrow p(\vec{z})=\prod_{k=1}^K \pi_k^{z_k}
\end{align*}

Conditional probability of $\vec{x}$ given a particular value for latent variable $\vec{z}$:
\begin{align*}
	p(\vec{x}\vert \vec{z}) = \prod_{k=1}^K \mathcal{N}(\vec{x}\vert \vec{\mu}_k, \vec{\Sigma}_k)^{z_k}
\end{align*}

Using Bayes theorem and marginalize the latent variable $\vec{z}$
\begin{align*}
	p(\vec{x}) =\sum_{\vec{z}}p(\vec{z})p(\vec{x}\vert \vec{z}) = \sum_{k=1}^K \pi_k\mathcal{N}(\vec{x}\vert \vec{\mu}_k, \vec{\Sigma}_k)
\end{align*}

Similarly, the posterior probability of $\vec{z}$ is given by
\begin{align*}
	\gamma (z_k) = \frac{p(\vec{x}\vert z_k=1)p(z_k=1)}{p(\vec{x})} = \frac{\pi_k \mathcal{N}(\vec{x}\vert \vec{\mu}_k, \vec{\Sigma}_k) }{\sum_{j=1}^K \pi_j\mathcal{N}(\vec{x}\vert \vec{\mu}_j, \vec{\Sigma}_j)}
\end{align*}

\subsubsection{Maximum Likelihood}
The log-likelihood function is given by
\begin{align*}
	\ln p(\vec{X} \vert \vec{\pi}, \vec{\mu}, \vec{\Sigma})&=\sum_{n=1}^N \ln\left(\sum_{k=1}^K \pi_k\mathcal{N}(\vec{x}\vert \vec{\mu}_k, 	\vec{\Sigma}_k)\right)
\end{align*}

The maximum likelihood method will yield the following:
\begin{align*}
	\vec{\mu}_k & = \frac{\sum_{n=1}^N \gamma(z_{nk})\vec{x}_n}{\sum_{n=1}^N \gamma(z_{nk})}\\
	\vec{\Sigma}_k& = \frac{\sum_{n=1}^N \gamma(z_{nk})(\vec{x}_n-\vec{\mu}_k)(\vec{x}_n-\vec{\mu}_k)^\top}{\sum_{n=1}^N \gamma(z_{nk})}\\
	\pi_k&= \frac{\sum_{n=1}^N \gamma(z_{nk})}{N}
\end{align*}

Issues with maximum likelihood:
\begin{enumerate}
	\item Presence of singularities: When we have at least two components in the mixture, one of them can have a finite variance and assign finite probability to all the data points, while the other component can shrink onto one specific data point and therefore contribute to an ever increasing additive value to the log likelihood.
	\item Identifiability: Solutions may not be unique, hence it may be hard to interpret the parameter values discovered by a model.
	\item The log likelihood equation is difficult to optimize over.
\end{enumerate}


\subsubsection{Expectation Maximization Formulation}
Supposed that in addition to $\vec{X}$, we were also given the values of the latent variables $\vec{Z}$, the likelihood and the log likelihood function are given by
\begin{align*}
	p(\vec{X}, \vec{Z}\vert \vec{\mu}, \vec{\Sigma}, \vec{\pi})& =\prod_{n=1}^{N}\prod_{k=1}^{K} \pi_k^{z_{nk}}\mathcal{N}(\vec{x}_n\vert \vec{\mu}_k, \vec{\Sigma}_k)^{z_{nk}}\\
	\ln p(\vec{X}, \vec{Z}\vert \vec{\mu}, \vec{\Sigma}, \vec{\pi})& =\sum_{n=1}^{N}\sum_{k=1}^{K} z_{nk}\left(\ln \pi_k + \ln \mathcal{N}(\vec{x}_n\vert \vec{\mu}_k, \vec{\Sigma}_k)\right)
\end{align*}

\begin{enumerate}
\item \textbf{Expectation Step}:

Taking the expectation on the log likelihood 
\begin{align*}
\mathbb{E}_{p(\vec{Z}\vert \vec{X}, \vec{\theta})}\left[\ln p(\vec{X}, \vec{Z}\vert \vec{\mu}, \vec{\Sigma}, \vec{\pi})\right]
& =\sum_{n=1}^{N}\sum_{k=1}^{K} \mathbb{E}_{p(\vec{Z}\vert \vec{X}, \vec{\theta})}[z_{nk}]\left(\ln \pi_k + \ln \mathcal{N}(\vec{x}_n\vert \vec{\mu}_k, \vec{\Sigma}_k)\right)\\
& = G(\vec{\theta})
\end{align*}

\begin{align*}
p(\vec{Z}\vert \vec{X}, \vec{\theta})\propto &
\prod_{n=1}^N\prod_{k=1}^K [\pi_k\mathcal{N}(\vec{x}_n\vert \vec{\mu}_k, \vec{\Sigma}_k)]^{z_{nk}}
\end{align*}

\begin{align*}
\mathbb{E}_{p(\vec{Z}\vert \vec{X}, \vec{\theta})}[z_{nk}]
&= \frac{\sum_{z_{nk}}z_{nk}[\pi_k\mathcal{N}(\vec{x}_n\vert \vec{\mu}_k, \vec{\Sigma}_k)]^{z_{nk}} }{\sum_{j=1}^K \pi_j \mathcal{N}(\vec{x}_n \vert \vec{\mu}_j, \vec{\Sigma}_j)}\\
&= \frac{\pi_k \mathcal{N}(\vec{x}_n \vert \vec{\mu}_k, \vec{\Sigma}_k)}{\sum_{j=1}^K \pi_j \mathcal{N}(\vec{x}_n \vert \vec{\mu}_j, \vec{\Sigma}_j)}\\
&=\gamma({z_{nk}})
\end{align*}

\item \textbf{Maximization Step}:
By taking the derivative of $(\vec{\theta})$ w.r.t $\vec{\theta}$ and set them to $0$, the parameters can be found to be
\begin{align*}
\vec{\mu}_k & = \frac{\sum_{n=1}^N \gamma(z_{nk})\vec{x}_n}{\sum_{n=1}^N \gamma(z_{nk})}\\
\vec{\Sigma}_k& = \frac{\sum_{n=1}^N \gamma(z_{nk})(\vec{x}_n-\vec{\mu}_k)(\vec{x}_n-\vec{\mu}_k)^\top}{\sum_{n=1}^N \gamma(z_{nk})}\\
\pi_k&= \frac{\sum_{n=1}^N \gamma(z_{nk})}{N}
\end{align*}

\end{enumerate}

\subsubsection{Expectation Maximization Algorithm}
Given a Gaussian mixture model, the goal is to maximize the likelihood functions w.r.t to the parameters:
\begin{enumerate}
	\item Initialize the means $\vec{\mu}_k$, covariances $\vec{\Sigma}_k$ and mixing coefficients $\pi_k$ and evaluate the initial of the log likelihood.
	\item E Step: Evaluate the responsibilities using the current parameter values
		\begin{align*}
			\gamma (z_k) = \frac{p(\vec{x}\vert \vec{z})p(\vec{z})}{p(\vec{x})} = \frac{\pi_k \mathcal{N}(\vec{x}\vert \vec{\mu}_k, \vec{\Sigma}_k) }	{\sum_{j=1}^K \pi_j\mathcal{N}(\vec{x}\vert \vec{\mu}_j, \vec{\Sigma}_j)}
		\end{align*}
	\item M Step: Re-estimate the parameters using the current responsibilities
		\begin{align*}
			\vec{\mu}_k^{new}&=\frac{1}{N_k}\sum_{n=1}^N \gamma(z_{nk})\vec{x}_n\\
			\vec{\Sigma}_k^{new}&=\frac{1}{N_k}\sum_{n=1}^N \gamma(z_{nk})(\vec{x}_n-\vec{\mu}_k)(\vec{x}_n-\vec{\mu}_k)^\top\\
			\pi_k^{new}&=\frac{N_k}{N}
		\end{align*}
	where
		\begin{align*}
			N_k & = \sum_{n=1}^N \gamma(z_{nk})
		\end{align*}
	\item Evaluate the log likelihood
		\begin{align*}
			\ln p(\vec{X} \vert \vec{\pi}, \vec{\mu}, \vec{\Sigma})&=\sum_{n=1}^N \ln\left(\sum_{k=1}^K \pi_k\mathcal{N}(\vec{x}\vert \vec{\mu}_k, 		\vec{\Sigma}_k)\right)
	\end{align*}
	and check for convergence of either parameter of the log likelihood. If the convergence criterion is not satisfied, return to step 2
\end{enumerate}


\subsection{Bernoulli Mixture Models}

\begin{align*}
p(\vec{x}\vert \vec{\mu}) & = \prod_{i=1}^D \mu_i^{x_i}(1-\mu_i)^{1-x_i}\\
\mathbb{E}[\vec{x}] & = \vec{\mu}\\
\mathbb{V}[\vec{x}] &= diag\lbrace \mu_i(1-\mu_i)\rbrace
\end{align*}

The mixture of the Bernoulli distributions is given by
\begin{align*}
p(\vec{x}\vert \vec{\mu}, \vec{\pi})&=\sum_{k=1}^K \pi_kp(\vec{x}\vert \vec{\mu}_k)\\
p(\vec{x}\vert \vec{\mu}_k) &= \prod^{D}_{i=1} \mu_{ki}^{x_i} (1- \mu_{ki}^{x_i})^{1-x_i}\\
\mathbb{E}[\vec{x}] & = \sum_{k=1}^K \pi_k\vec{\mu}_k\\
\mathbb{V}[\vec{x}] &= \sum_{k=1}^K \pi_k \left(\vec{\Sigma}_k + \vec{\mu}_k\vec{\mu}_k^\top\right) - \mathbb{E}[\vec{X}]\mathbb{E}[\vec{X}]^\top\\
\vec{\Sigma}_k & = diag\lbrace \mu_{ki}(1-\mu_{ki})\rbrace
\end{align*}

Let $\vec{z}$ be the one hot representation, we have
\begin{align*}
p(\vec{x} \vert \vec{z}, \vec{\mu}) &= \prod_{k=1}^K p(\vec{x}\vert \vec{\mu}_k)^{z_k}\\
p(\vec{z}\vert \vec{\pi}) &= \prod_{k=1}^K \pi_k^{z_k}
\end{align*}

The log likelihood function is given by
\begin{align*}
\ln p(\vec{X}, \vec{Z} \vert \vec{\mu}, \vec{\pi})&=\sum_{n=1}^N\sum_{k=1}^Kz_{nk}
\left\lbrace 
\ln \pi_k +\sum_{i=1}^D[x_{ni}\ln\mu_{ki}+(1-x_{ni})\ln(1-\mu_{ki})]
\right\rbrace
\end{align*}

The E step: we then take the expectation of the log likelihood above
\begin{align*}
\mathbb{E}_{\vec{Z}}[\ln p(\vec{X}, \vec{Z} \vert \vec{\mu}, \vec{\pi})]&=\sum_{n=1}^N\sum_{k=1}^K\gamma(z_{nk})
\left\lbrace 
\ln \pi_k +\sum_{i=1}^D[x_{ni}\ln\mu_{ki}+(1-x_{ni})\ln(1-\mu_{ki})]
\right\rbrace\\
&\\
\gamma(z_{nk})& = \frac{\sum_{z_{nk}} z_{nk}[\pi_kp(\vec{x}_n\vert \vec{\mu}_k)]^{z_{nk}} }{\sum_{z_{nj}}[\pi_jp(\vec{x}_n\vert \vec{\mu}_j)]^{z_{nj}}}
=\frac{\pi_kp(\vec{x}_n\vert \vec{\mu}_k)}{\sum_{j=1}^K \pi_jp(\vec{x)_n}\vert \vec{\mu}_j)}
\end{align*}

The M step: we then maximize the log likelihood wrt to the parameters
\begin{align*}
\vec{\mu}_k &=\frac{1}{N_k}\sum_{n=1}^N \gamma(z_{nk})\\
\pi_k&=\frac{N_k}{N}\\
N_k &= \sum_{n=1}^N \gamma(z_{nk})
\end{align*}

\subsection{Convergence of the EM Algorithm}

\begin{align*}
p(\vec{X} \vert \vec{\theta}) = \sum_{\vec{Z}} p(\vec{X},\vec{Z} \vert \vec{\theta}) &\Rightarrow \ln p(\vec{X} \vert \vec{\theta}) = \mathcal{L}(q, \vec{\theta}) + KL(q\parallel p)\\
\mathcal{L}(q, \vec{\theta})&=\sum_{\vec{Z}} q(\vec{Z})\ln \left[\frac{p(\vec{X}, \vec{Z}\vert \vec{\theta})}{q(\vec{Z})}\right]\\
 KL(q\parallel p)&=-\sum_{\vec{Z}} q(\vec{Z})\ln \left[\frac{p(\vec{Z}\vert \vec{X}, \vec{\theta})}{q(\vec{Z})}\right]
\end{align*}

In the E step, the lower bound $\mathcal{L}(q,\vec{\theta}^{old})$ is maximized with respect to $q(\vec{Z})$. The solution to this maximization problem can be done by setting $q(\vec{Z}) = p (\vec{Z}\vert \vec{X}, \vec{\theta}^{old})$ such that $KL(q\parallel p)=0$.\\

In the M step, the lower bound $\mathcal{L}(q,\vec{\theta}^{old})$ is maximized with respect to $\vec{\theta}$. However, the KL divergence will no longer be zero. Hence, the increase in the upper bound of the log likelihood function is greater than the increase in the lower bound.\\

By substituting $q(\vec{Z}) = p(\vec{Z}\vert\vec{X}, \vec{\theta}^{old})$, we get the lower bound to be the following after the E step:
\begin{align*}
\mathcal{L}(q,\vec{\theta}) & = \sum_{\vec{Z}}p(\vec{Z}\vert \vec{X}, \vec{\theta}^{old})\ln p(\vec{X},\vec{Z}\vert \vec{\theta}) - \sum_{\vec{Z}}p(\vec{Z}\vert \vec{X}, \vec{\theta}^{old})\ln p(\vec{Z} \vert \vec{X}, \vec{\theta}^{old})\\
&= \mathcal{Q}(\vec{\theta},\vec{\theta}^{old} ) + const 
\end{align*}


\newpage

\section{Probabilistic PCA}

\begin{align*}
\vec{x} &= \vec{Wy} + \vec{\mu} + \vec{e}\\
\vec{e} & \sim \mathcal{N}(\vec{e}\vert \vec{0}, \sigma^2\vec{I})\\
\vec{y} & \sim \mathcal{N}(\vec{y}\vert \vec{0}, \vec{I})
\end{align*}


\begin{figure}[H]
\begin{center}
\includegraphics[width = 0.4\hsize]{./figures/PPCA.png} % this includes the figure and specifies that it should span 0.7 times the horizontal size of the 
\caption{Gaussian Mixture Model.} % caption of the figure
\label{fig:PPCA} % a label. When we refer to this label from the text, the figure number is included automatically
\end{center}
\end{figure}

\subsection{Maximum Likelihood}
\begin{enumerate}
\item We are interested in the distribution of $\vec{x}$ and the parameters $\vec{\theta}$. The marginal distribution of $\vec{x}$ can be obtained by integrating $\vec{y}$ out of the joint distribution.
\begin{align*}
p(\vec{x}_1,\dots, \vec{x}_N \vert \vec{\theta})
&=\int_{\vec{y_i}}p(\vec{x}_1,\dots, \vec{x}_N, \vec{y}_1,\dots, \vec{y}_N \vert \vec{\theta}) d\vec{y}_1\dots d\vec{y}_N\\
&= \int_{\vec{y_i}}\prod_{i=1}^N p(\vec{x}_i \vert \vec{y}_i,\vec{W}, \vec{\mu}, \sigma)p(\vec{y}_i)d\vec{y}_1\dots d\vec{y}_N\\
&= \prod_{i=1}^N\int_{\vec{y_i}} p(\vec{x}_i \vert \vec{y}_i,\vec{W}, \vec{\mu}, \sigma)p(\vec{y}_i)d\vec{y}_i
\end{align*}

\item The joint distribution can be obtained by applying Bayes' Rule.
\begin{align*}
p(\vec{x}_1,\dots, \vec{x}_N, \vec{y}_1,\dots, \vec{y}_N \vert \vec{\theta})
& = \prod_{i=1}^N p(\vec{x}_i \vert \vec{y}_i,\vec{W}, \vec{\mu}, \sigma)p(\vec{y}_i)\\
p(\vec{x}_i \vert \vec{y}_i,\vec{W}, \vec{\mu}, \sigma)p(\vec{y}_i) 
&= \left[(2\pi\sigma^2)^{-\frac{F}{2}} e^{-\frac{1}{2\sigma^2}(\vec{x}_i-\vec{\mu}-\vec{Wy}_i)^\top(\vec{x}_i-\vec{\mu}-\vec{Wy}_i)}\right]\left[(2\pi)^{-\frac{d}{2}}e^{-\frac{1}{2}\vec{y}_i^\top\vec{y}_i}\right]
\end{align*}

\item By collecting the $\vec{y}$ in the exponents, completing the squares and then applying the Woodbury identity (more details please refer to notes in 496):
\begin{align*}
p(\vec{x}_i\vert \vec{W}, \vec{\mu}, \sigma^2) & = \mathcal{N}(\vec{x}_i\vert \vec{\mu}, \vec{D} )\\
\vec{D}&=\vec{WW}^\top +\sigma^2\vec{I}\\
p(\vec{y}_i\vert \vec{x}_i, \vec{W}, \vec{\mu}, \sigma^2) & = \mathcal{N}(\vec{y}_i \vert \vec{M}^{-1}\vec{W}^\top(\vec{x}_i-\vec{\mu}), \sigma^2\vec{M}^{-1})\\
\vec{M}&=\sigma^2\vec{I} + \vec{W^\top W}
\end{align*}

\item  Maximizing the likelihood and solve for the parameters
\begin{align*}
\vec{S_t} & = \vec{U\Lambda U}^\top \text{ ($\vec{S_t}$ is the covariance matrix)}\\
\sigma^2 & = \frac{1}{F-d} \sum_{j= d+1}^F \lambda_j\\
\vec{W_d}&= \vec{U_d}(\vec{\Lambda} - \sigma^2\vec{I})^{\frac{1}{2}}\vec{V}^\top
\end{align*}

\item Hence we no longer have a projection but:
\begin{align*}
\mathbb{E}_{\vec{p(\vec{y}_i\vert \vec{x}_i)}} [\vec{y}_i] & = \vec{M}^{-1}\vec{W}^\top (\vec{x}_i - \vec{\mu})\\
\vec{\widehat{x}_i}&= \vec{W}\mathbb{E}_{\vec{p(\vec{y}_i\vert \vec{x}_i)}} [\vec{y}_i]  + \vec{\mu}
\end{align*}

\end{enumerate}


\subsection{EM PPCA}
\begin{enumerate}
\item Write the log likelihood of the joint distribution of the observed and latent variables
\begin{align*}
p(\vec{X,Y} \vert \vec{\theta}) &= \prod_{i=1}^N p(\vec{x}_i \vert\vec{y}_i, \vec{\theta})p(\vec{y}_i)\\
\ln p(\vec{X,Y} \vert \vec{\theta}) &= \sum_{i=1}^N \left(\ln p(\vec{x}_i \vert\vec{y}_i,\vec{\theta})+ \ln p(\vec{y}_i)\right)\\
\ln p(\vec{x}_i \vert \vec{y}_i, \vec{\theta}) &= -\frac{F}{2}\ln (2\pi\sigma^2) - \frac{1}{2\sigma^2}(\vec{x}_i-\vec{Wy}_i-\vec{\mu})^\top (\vec{x}_i-\vec{Wy}_i-\vec{\mu})\\
\ln p(\vec{y}_i)&= -\frac{1}{2}\vec{y}_i^\top \vec{y}_i-\frac{D}{2}\ln 2\pi
\end{align*}

\item \textbf{E-Step:} Take the expectation on the log likelihood on the joint distribution:
\begin{align*}
\ln p(\vec{X,Y} \vert \vec{\theta}) &=\sum_{i=1}^N\left[-\frac{F}{2}\ln 2\pi\sigma^2 -\frac{1}{2\sigma^2}(\vec{x}_i-\vec{Wy}_i-\vec{\mu})^\top (\vec{x}_i-\vec{Wy}_i-\vec{\mu}) -\frac{1}{2}\vec{y}_i^\top\vec{y}_i-\frac{D}{2}\ln 2\pi\right]
\end{align*}

Expanding the above and use the identities $\text{tr}\left[\vec{y}_i^\top \vec{WW}^\top \vec{y}_i\right]=\text{tr}\left[\vec{y}_i\vec{y}_i^\top \vec{WW}^\top \right]$
\begin{align*}
\mathbb{E}_{p(\vec{Y\vert X})} [\ln p(\vec{X,Y} \vert \vec{\theta})]
&=-\frac{NF}{2}\ln 2\pi\sigma^2 -\frac{ND}{2}\ln 2\pi\\
& -\sum_{i=1}^N\left\lbrace\frac{1}{2\sigma^2}\left[(\vec{x}_i-\vec{\mu})^\top (\vec{x}_i-\vec{\mu}) -2 (\vec{x}_i-\vec{\mu})^\top \vec{W} \mathbb{E}[\vec{y}_i] \right.\right.\\
&+\left.\left.\text{tr}\left[\mathbb{E}(\vec{y}_i\vec{y}_i^\top)\vec{W^\top W}\right]\right]+\frac{1}{2}\text{tr}\left[\mathbb{E}(\vec{y}_i^\top \vec{y}_i)\right]\right\rbrace
\end{align*}

We can obtain the moments of $\vec{y_i}$ from the earlier derivation of the PPCA
\begin{align*}
p(\vec{y}_i\vert \vec{x}_i, \vec{W}, \vec{\mu}, \sigma^2) & = \mathcal{N}(\vec{y}_i \vert \vec{M}^{-1}\vec{W}^\top(\vec{x}_i-\vec{\mu}), \sigma^2\vec{M}^{-1})\\
\vec{M}&=\sigma^2\vec{I} + \vec{W^\top W}\\
\mathbb{E}_{p(\vec{Y\vert X})}[\vec{y}_i]&=\vec{M}^{-1}\vec{W}^\top(\vec{x}_i-\vec{\mu})\\
\mathbb{E}_{p(\vec{Y\vert X})}[\vec{y}_i \vec{y}_i^\top]&=\sigma^2\vec{M}^{-1} +\mathbb{E}[\vec{y}_i]\mathbb{E}[\vec{y}_i]^\top
\end{align*}

\item \textbf{M-Step:} Maximize the log likelihood
\begin{align*}
\vec{\mu}&= \frac{1}{N}\sum_{i=1}^N(\vec{x}_i - \vec{W}\mathbb{E}[\vec{y}_i])\\
\vec{W}&=\left[\sum_{i=1}^N (\vec{x}_i - \vec{\mu})\mathbb{E}[\vec{y}_i]^\top]\right]\left[\sum_{i=1}^N \mathbb{E}[\vec{y}_i\vec{y}_i^\top]\right]^{-1}\\
\sigma^2&=\frac{1}{NF}\sum_{i=1}^N\left\lbrace \parallel \vec{x_i-\mu} \parallel^2 -2\mathbb{E}[\vec{y}_i]^\top\vec{W^\top (x_i-\mu)} +\text{tr}(\mathbb{E}[\vec{y}_i\vec{y}_i^\top]\vec{W^\top W})\right\rbrace
\end{align*}


\item Advantages of EM PPCA:
\begin{itemize}
\item A complexity of $O(NFD)$ can be significantly smaller than $O(NF^2)$ (from the computation of the covariance)
\item The EM procedure can be extended to factor analysis model
\item EM allows us to deal with missing values
\end{itemize}

\end{enumerate}


\section{PPCA Mixture Model}

\begin{figure}[H]
\begin{center}
\includegraphics[width = 0.4\hsize]{./figures/PPCAMixture.png} % this includes the figure and specifies that it should span 0.7 times the horizontal size of the 
\caption{PPCA Mixture Model.} % caption of the figure
\label{fig:PPCA Mixture Model} % a label. When we refer to this label from the text, the figure number is included automatically
\end{center}
\end{figure}


The model is given by:
\begin{align*}
	\vec{x}_i &= \vec{\mu}_k + \vec{W}_k\vec{y}_{ik} + \vec{e}_{ik}\\
	\vec{y}_{ik} &\sim \mathcal{N} (\vec{y}_{ik}\vert \vec{0,I})\\
	\vec{e}_{ik} & \sim \mathcal{N} (\vec{e}_{ik}\vert \vec{0},\sigma^2_k\vec{I}), \forall k = 1,\dots, K\\
	p(\vec{x}_i\vert  \vec{z}_{ik} &=1, \vec{y}_{ik}, \vec{\mu_k}, \vec{W_k}, \sigma^2_k \vec{I}) = \mathcal{N}(\vec{x}_i \vert \vec{\mu}_k + \vec{W}_k\vec{y}_{ik})\\
	p(\vec{x}_i\vert \vec{\theta})& = \sum_{k=1}^K \pi_k \mathcal{N}(\vec{x}_i \vert \vec{\mu}_k, \vec{D}_k)\\\
	\vec{D}_k &= \vec{W}_k\vec{W}_k^\top + \sigma_k^2\vec{I}
\end{align*}


\subsection{Probability Distributions for EM Formulation}

\begin{enumerate}
\item Priors distributions:
	\begin{align*}
		p(\vec{z}_i \vert \vec{\theta}_z) & = \prod_{k=1}^K \pi_k^{z_{ik}}\\
		p(\vec{Y}_i \vert \vec{z}_i, \vec{\theta}_z) & = \prod_{k=1}^K p(y_{ik})^{z_{ik}} = \prod_{k=1}^K \mathcal{N}(y_{ik}\vert \vec{0}, \vec{I})^{z_{ik}}
	\end{align*}

\item Conditional distribution:
	\begin{align*}
		p(\vec{x}_i \vert \vec{z}_i, \vec{Y}_i, \vec{\theta_x}) 
		& = \prod_{k=1}^K p(\vec{x}_i \vert \vec{z}_{ik}=1, \vec{y}_{ik}, \vec{W}_k, \vec{\mu}_k, \vec{\sigma}_k^2\vec{I})^{z_{ik}}\\
		& = \prod_{k=1}^K \mathcal{N}(\vec{x}_{ik}\vert \vec{W}_k \vec{y}_{ik}+\vec{\mu}_k, \sigma_k\vec{I})^{z_{ik}}		
	\end{align*}

\item Marginal distributions per cluster is found by integrating $\vec{y}$ out:
	\begin{align*}
		p(\vec{x}_i \vert \vec{z}_{ik}=1, \vec{W}_k, \vec{\mu}_k, \sigma_k^2) = \mathcal{N}(\vec{x}_i\vert \vec{\mu}_k, \vec{D}_k)\\
		\vec{D}_k = \vec{W}_k\vec{W}_k^\top + \sigma_k^2 \vec{I}
	\end{align*}

\item Full marginal distributions:
	\begin{align*}
		p(\vec{x}_i \vert \vec{\theta}) 
		&= \sum_{k=1}^K p(z_{ik}=1)p(\vec{x}\vert z_{ik}=1, \theta_x)\\
		&= \sum_{k=1}^K \pi_k \mathcal{N}(\vec{x}_i\vert \vec{\mu}_k, \vec{D}_k)		
	\end{align*}

\item Posteriors on $\vec{y}_{ik}$ (we will take expectation on this in the E step):
	\begin{align*}
		p(\vec{y}_{ik} \vert \vec{x}_{ik}, z_{ik}=1, \vec{W}_{k}, \vec{\mu}_{k}, \sigma_k^2 )
		&= \mathcal{N}\left(\vec{y}_{ik}\vert \vec{M}_k^{-1}\vec{W}_k^\top(\vec{x}_i - \vec{\mu}_k),\sigma_k^2 \vec{M}_k^{-1}\right)\\
		\mathbb{E}(\vec{y}_{ik})& = \vec{M}_k^{-1}\vec{W}_k^\top (\vec{x}_i - \vec{\mu}_k)\\
		\mathbb{E}(\vec{y}_{ik}\vec{y}_{ik}^\top)& =\sigma_k^2\vec{M}_k^{-1}+\mathbb{E}(\vec{y}_{ik})\mathbb{E}(\vec{y}_{ik})^\top
	\end{align*}

\item Posteriors on $\vec{z}_i$  (we will take expectation on this in the E step):
\begin{align*}
p(\vec{z}_i \vert \vec{x}_i, \theta)
&= \frac{p(\vec{x}_i, \vec{z}_i \vert \theta)}{p(\vec{x}_i \vert \theta)}
= \frac{p(\vec{x}_i\vert \vec{z}_i, \theta)p(\vec{z}_i \vert \theta)}{p(\vec{x}_i \vert \theta)}
= \frac{\prod_k^K \left(\mathcal{N}(\vec{x}_i \vert \vec{\mu}_k, \vec{D}_k)\pi_k\right)^{z_{nk}}}{\sum_{l =1}^K \pi_l \mathcal{N}(\vec{x}_i \vert \vec{D}_l, \vec{\mu}_l)}\\
\mathbb{E}[z_{ik}] & = \sum_{z_{ik}=0,1} z_{ik} p(\vec{z_i}\vert \vec{x}_i, \theta)
= \frac{\pi_k\mathcal{N}(\vec{x}_i \vert \vec{\mu}_k, \vec{D}_k)}{\sum_{l=1}^K \pi_l\mathcal{N}(\vec{x}_i \vert \vec{\mu}_l, \vec{D}_l)}
\end{align*}

\end{enumerate}


\subsection{Formulation of EM}
\begin{enumerate}
\item Setting the joint likelihood (decomposition below can be guided by graphical model):	
	\begin{align*}
	p(\vec{X}, \vec{Y}, \vec{Z} \vert \theta) 
	&= p(\vec{X} \vert \vec{Z}, \vec{Y}, \theta_x)p(\vec{Z}, \vec{Y}\vert \theta_Z)
	= p(\vec{X} \vert \vec{Z}, \vec{Y}, \theta_x)p(\vec{Y}\vert \vec{Z})p(\vec{Z}\vert \theta_Z)\\
	p(\vec{Z}\vert \theta_Z)
	&= \prod_{i=1}^N \prod_{k=1}^K \pi_k^{z_{ik}}\\
	p(\vec{Y}\vert \vec{Z})
	&= \prod_{i=1}^N p(\vec{Y}_i\vert \vec{z}_i, \theta_Z)
	= \prod_{i=1}^N \prod_{k=1}^3 p(\vec{y}_{ik})^{z_{ik}}
	= \prod_{i=1}^N \prod_{k=1}^3 \mathcal{N}(\vec{y}_{ik}\vert \vec{0}, \vec{I})^{z_{ik}}\\
	p(\vec{X} \vert \vec{Z}, \vec{Y}, \theta_x)
	&=\prod_{i=1}^N p(\vec{x}_i \vec{z}_i, \vec{Y}_i, \theta_x
	=\prod_{i=1}^N\prod_{k=1}^K \mathcal{N}(\vec{x}_i \vert \vec{W}_k\vec{y}_{ik}+\vec{\mu}_k, \sigma_k^2)^{z_{ik}}
	\end{align*}
	
	
	
Hence the log likelihood of the joint distribution:
\begin{align*}
\ln p(\vec{X}, \vec{Z}, \vec{Y} \vert \theta) 
&=\sum_{i=1}^N \sum_{k}^K z_{ik}\left[\ln  \mathcal{N}(\vec{x}_i \vert \vec{W}_k\vec{y}_{ik}+\vec{\mu}_k, \sigma_k^2) + \ln \mathcal{N}(\vec{y}_{ik}\vert \vec{0}, \vec{I}) + \ln\pi_k\right]\\
&= \sum_{i=1}^N \sum_{k}^K z_{ik}\left[-\frac{1}{2\sigma_k^2}(\vec{x}_i - \vec{\mu}_k - \vec{W}_k\vec{y}_k)^\top(\vec{x}_i - \vec{\mu}_k - \vec{W}_k\vec{y}_k)-\frac{F}{2}\ln 2\pi - F\ln\sigma_k\right]\\
&+ \sum_{i=1}^N\sum_{k}^K z_{ik}\left[-\frac{1}{2}\vec{y}_{ik}^\top\vec{y}_{ik} - \frac{d}{2}\ln 2\pi\right]+\sum_{i=1}^N\sum_{k}^K z_{ik}\ln \pi_k
\end{align*}

\item Taking the expectation yields
\begin{align*}
&\mathbb{E}[\ln p(\vec{X}, \vec{Z}, \vec{Y} \vert \theta) ]\\
&=-\frac{1}{2\sigma_k^2} \sum_{i=1}^N \sum_{k}^K \mathbb{E}[z_{ik}]\left[
\parallel \vec{x}_i -\vec{\mu}_k\parallel^2 - \mathbb{E}[\vec{y}_{ik}]^\top\vec{W}_k^\top(\vec{x}_i-\vec{\mu}_k) + \text{tr}(\vec{WW}^\top \mathbb{E}[\vec{y}_{ik}\vec{y}_{ik}^\top])\right]\\
&+\sum_{i=1}^N\sum_{k}^K\left[\frac{F}{2}\ln 2\pi - F\ln\sigma_k\right]+ \sum_{i=1}^N\sum_{k}^K \mathbb{E}[z_{ik}]\left[-\frac{1}{2}\text{tr}(\mathbb{E}[\vec{y}_{ik}\vec{y}_{ik}^\top]) - \frac{d}{2}\ln 2\pi\right]+\sum_{i=1}^N\sum_{k}^K \mathbb{E}[z_{ik}]\ln \pi_k
\end{align*}

\item The maximization step:
\begin{align*}
\pi_k &= \frac{1}{N}\sum^N_{i=1} \mathbb{E}[z_{ik}]\\
\vec{\mu}_k &= \frac{\sum_{i=1}^N \mathbb{E}[z_{ik}](\vec{x}_i - \vec{W}_k \mathbb{E}[\vec{y}_{ik}])} {\sum_{i=1}^N \mathbb{E}[z_{ik}]}\\
\vec{W}_k & = \left[\sum_{i=1}^N\mathbb{E}[z_{ik}](\vec{x}_i - \vec{\mu})\mathbb{E}[\vec{y}_{ik}^\top]\right]\left[\sum_{i=1}^N\mathbb{E}[z_{ik}]\mathbb{E}[\vec{y_{ik}}\vec{y}_{ik}^\top])  \right]^{-1}\\
\sigma_k^2 = \frac{1}{F\sum_{i=1}^N \mathbb{E}[z_{ik}]}&\sum_{i=1}^N \mathbb{E}[z_{ik}]\left[\parallel\vec{x}_i-\vec{\mu}_k\parallel^2 - \mathbb{E}[\vec{y}_{ik}]^\top \vec{W}^\top_k(\vec{x}_i-\vec{\mu}_k) + \text{tr}\left(\vec{W}_k^\top \vec{W}_k \mathbb{E}[\vec{y}_{ik}\vec{y}_{ik}^\top]\right)\right]
\end{align*}

\end{enumerate}

\newpage

\section{PPCA with Missing Values}
Given the PPCA model:
\begin{align*}
\vec{x} &= \vec{Wy} + \vec{\mu} + \vec{e}\\
\vec{e} & \sim \mathcal{N}(\vec{e}\vert \vec{0}, \sigma^2\vec{I})\\
\vec{y} & \sim \mathcal{N}(\vec{y}\vert \vec{0}, \vec{I})
\end{align*}
\begin{align*}
p(\vec{x}_i \vert \vec{W}, \vec{\mu}, \sigma^2)&= \mathcal{N}(\vec{x}_i\vert \vec{\mu}, \vec{D})\\
p(\vec{y}_i \vert \vec{x}_i, \vec{W}, \vec{\mu}, \sigma^2)&=\mathcal{N}(\vec{y}_i \vert \vec{M}^{-1}\vec{W}^\top (\vec{x}_i-\vec{\mu}), \sigma^2\vec{M}^{-1})\\
\vec{D} &= \vec{WW}^\top +\sigma^2\vec{I} \\
\vec{M} &= \vec{W}^\top \vec{W} +\sigma^2\vec{I}
\end{align*}

\begin{enumerate}

	\item We can reformulate the PPCA model to take into account of missing data:
		\begin{align*}
			\vec{x} = 	\begin{bmatrix}
								\vec{x}^o \\ \vec{x}^u
							\end{bmatrix}
			&\Rightarrow
			p\left(\vec{x}^o, \vec{x}^u\right) = \mathcal{N}\left(\left.
				\begin{bmatrix}
					\vec{x}^o \\ \vec{x}^u
				\end{bmatrix}
			\right\vert 
				\begin{bmatrix}
					\vec{\mu}^o \\ \vec{\mu}^u
				\end{bmatrix},
				\begin{bmatrix}
					\vec{D}^{oo} 		&\vec{D}^{ou} 		\\ 
					\vec{D}^{uo}			&\vec{D}^{uu}
				\end{bmatrix}\right)\\
				p(\vec{x}^u\vert \vec{x}^o)
				&=\mathcal{N}\left(\vec{x}^u\left \vert\vec{\mu}^u +\vec{D}_{uo}\vec{D}_{oo}^{-1}(\vec{x}^o - \vec{\mu}^o), \vec{D}_{uu}-\vec{D}_{uo}\vec{D}_{oo}^{-1}\vec{D}_{ou}  				\right.\right )
\end{align*}

	\item For convenience, we can rewrite this formulation as derive its first order and second order moments:
		\begin{align*}
			p(\vec{x}_i\vert \vec{x}_i^o)  &= \mathcal{N} \left(\vec{x}_i\left\vert \vec{z}_i, \vec{Q}\right. \right)\\
			\vec{z}_i &=\begin{bmatrix}
			\vec{x}_i^o\\
			\vec{\mu}^u +\vec{D}_{uo}\vec{D}_{oo}^{-1}(\vec{x}^o - \vec{\mu}^o)
			\end{bmatrix}\\
			\vec{Q} & = \begin{bmatrix}
			0	&	0\\
			0	& 	\vec{D}_{uu}-\vec{D}_{uo}\vec{D}_{oo}^{-1}\vec{D}_{ou} 
			\end{bmatrix}
		\end{align*}		
		
		\begin{align*}
			\mathbb{E}_{p(\vec{x}_i\vert \vec{x}_i^o)}[\vec{x}_i] &=\vec{z}_i\\
			\mathbb{E}_{p(\vec{x}_i\vert \vec{x}_i^o)}[\vec{x}_i\vec{x}_i^\top] &=\vec{Q}+\vec{z}_i\vec{z}_i^\top\\
			\mathbb{E}_{p(\vec{x}_i\vert \vec{x}_i^o)}[(\vec{x}_i-\vec{a})(\vec{x}_i-\vec{a})^\top] &=\vec{Q}+(\vec{z}_i-\vec{a})(\vec{z}_i-\vec{a})^\top\\ 
		\end{align*}
		
	\item Writing down the log likelihood for the observed variable $X$ and the latent variable $Y$
		\begin{align*}
			\ln p(\vec{X}, \vec{Y} \vert \vec{\theta})
			&= -\frac{1}{2\sigma^2}\sum_{i=1}^N \left\lbrace\text{tr}\left[(\vec{x}_i-\vec{\mu})(\vec{x}_i-\vec{\mu})^\top\right]-  2\text{tr} \left[\vec{y}_i(\vec{x}_i-\vec{\mu})^\top \vec{W}\right]+\text{tr}\left[\vec{W}^\top\vec{Wy}_i\vec{y}_i^\top\right]\right\rbrace\\
			&-\frac{1}{2}\sum_{i=1}^N \text{tr}[\vec{y}_i\vec{y}_i^\top]-\frac{NF}{2} \ln 2\pi - NF \ln \sigma - N \ln 2\pi
		\end{align*}
	
	\item Take the expectation with regards to the probability distribution $p(\vec{X}, \vec{Y} \vert \vec{X}^o)$
		\begin{align*}
			&\mathbb{E}_{p(\vec{x}_i,\vec{y}_i\vert \vec{x}_i^o)}\left[\ln p(\vec{X}, \vec{Y} \vert \vec{\theta})\right]\\
			&= -\frac{1}{2\sigma^2}\sum_{i=1}^N \left\lbrace\text{tr}\left(\mathbb{E}\left[(\vec{x}_i-\vec{\mu})(\vec{x}_i-\vec{\mu})^\top\right]\right)-  2\text{tr} \left(\mathbb{E}\left[\vec{y}_i(\vec{x}_i-\vec{\mu})^\top\right] \vec{W}\right)+\text{tr}\left(\vec{W}^\top\vec{W}\mathbb{E}\left[\vec{y}_i\vec{y}_i^\top\right]\right)\right\rbrace\\
			&-\frac{1}{2}\sum_{i=1}^N \text{tr}\left(\mathbb{E}\left[\vec{y}_i\vec{y}_i^\top\right]\right)-\frac{NF}{2} \ln 2\pi - NF \ln \sigma - N \ln 2\pi
		\end{align*}
		
		\begin{align*}
			\mathbb{E}_{p(\vec{x}_i,\vec{y}_i\vert \vec{x}_i^o)}[(\vec{x}_i-\vec{\mu})(\vec{x}_i-\vec{\mu})^\top] 
			&=\mathbb{E}_{p(\vec{x}_i,\vec{y}_i\vert \vec{x}_i^o)}[(\vec{x}_i-\vec{\mu})(\vec{x}_i-\vec{\mu})^\top]\\ 
			&=\vec{Q}+(\vec{z}_i-\vec{\mu})(\vec{z}_i-\vec{\mu})^\top\\ 			
			\mathbb{E}_{p(\vec{x}_i,\vec{y}_i\vert \vec{x}_i^o)}[\vec{y}_i(\vec{x}_i-\vec{\mu})^\top] 
			&=\mathbb{E}_{p(\vec{x}_i\vert \vec{x}_i^o)}\left[\mathbb{E}_{p(\vec{y}_i\vert \vec{x}_i^o)} \left[\vec{y}_i(\vec{x}_i-\vec{\mu})^\top\right]\right] \\
			&=\mathbb{E}_{p(\vec{x}_i\vert \vec{x}_i^o)}\left[\vec{M}^{-1}\vec{W}^\top (\vec{x}_i-\vec{\mu})(\vec{x}_i-\vec{\mu})^\top\right]\\
			& =\vec{M}^{-1}\vec{W}^\top \mathbb{E}_{p(\vec{x}_i\vert \vec{x}_i^o)}\left[(\vec{x}_i-\vec{\mu})(\vec{x}_i-\vec{\mu})^\top\right]\\
			\mathbb{E}_{p(\vec{x}_i,\vec{y}_i\vert \vec{x}_i^o)}[\vec{y}_i\vec{y}_i^\top] 
			&=\mathbb{E}_{p(\vec{x}_i\vert \vec{x}_i^o)}\left[\mathbb{E}_{p(\vec{y}_i\vert \vec{x}_i^o)} \left[\vec{y}_i\vec{y}_i^\top\right]\right] \\
			&=\mathbb{E}_{p(\vec{x}_i\vert \vec{x}_i^o)}\left[\sigma^2 \vec{M}^{-1}+\vec{M}^{-1}\vec{W}^\top (\vec{x}_i-\vec{\mu})(\vec{x}_i-\vec{\mu})^\top\vec{W}\vec{M}^{-1}\right]\\
			& =\sigma^2\vec{M}^{-1}+ \vec{M}^{-1}\vec{W}^\top\mathbb{E}_{p(\vec{x}_i\vert \vec{x}_i^o)}\left[(\vec{x}_i-\vec{\mu})(\vec{x}_i-\vec{\mu})^\top\right] \vec{W} \vec{M}^{-1}
		\end{align*}

	\item Maximization step:
		\begin{align*}
				\vec{\mu}	&=\frac{1}{N}\sum_{i=1}^N \vec{z}_i\\
				\vec{W}	&=\left[\sum_{i=1}^N\mathbb{E}\left[(\vec{x}_i-\vec{\mu})\vec{y}_i^\top\right]\right]\left[\sum_{i=1}^N\mathbb{E}\left[\vec{y}_i\vec{y}_i^\top\right]\right]^{-1}\\
				\sigma^2 &=\frac{1}{NF}\sum_{i=1}^N\left\lbrace \text{tr}\left(\mathbb{E}\left[(\vec{x}_i-\vec{\mu})(\vec{x}_i-\vec{\mu})^\top\right]\right) 
				-2 \text{tr}\left( \mathbb{E} \left[ \vec{y}_i (\vec{x}_i-\vec{\mu})^\top \right]\vec{W} \right)
				+ \text{tr}\left(\mathbb{E}\left[\vec{y}_i\vec{y}_i^\top\right]\vec{W}^\top\vec{W}\right)	
				\right\rbrace\\
				\vec{D} & = \vec{W}\vec{W}^\top +\sigma^2\vec{I}
		\end{align*}			
	

\end{enumerate}

\newpage


\section {Hidden Markov Model}
\subsection{Markov Property and Markov Chains}

\begin{enumerate}
	\item Markov Property:
		\begin{align*}
			p(\vec{x_i} \vert \vec{x}_1, \ldots, \vec{x}_{i-1}) &= p(\vec{x}_i \vert \vec{x}_{i-1})
		\end{align*}
		
	\item Markov Model: Given the observations $\lbrace \vec{x}_t \rbrace_{t=1}^N$
		\begin{align*}
			p(\vec{x}_1, \ldots, \vec{x}_N) &= p(\vec{x}_1)\prod_{t=2}^N p(\vec{x}_t \vert \vec{x}_{t-1}) & \text{bigram model}\\
			p(\vec{x}_1, \ldots, \vec{x}_N) &= p(\vec{x}_1)p(\vec{x}_2\vert \vec{x}_1 )\prod_{t=3}^N p(\vec{x}_t \vert \vec{x}_{t-1},\vec{x}_{t-2}) & \text{trigram model}\\
			p(\vec{x}_1, \ldots, \vec{x}_N) &= p(\vec{x}_1)p(\vec{x}_2\vert \vec{x}_1 )p(\vec{x}_3\vert \vec{x}_1,\vec{x}_2 )\prod_{t=4}^N p(\vec{x}_t \vert \vec{x}_{t-1},\vec{x}_{t-2},\vec{x}_{t-3}) & \text{n-gram model}
		\end{align*}	
		
		\item A transition matrix $\vec{A}$ specifies the probabilities of getting from state $i$ (\textit{row}) to state $j$ (\textit{column}) in one step. Note that $\vec{A}$ is a stochastic matrix, i.e. the row must sum to 1.
		
		\item Stationary distribution is the long term distribution over the states
					\begin{align*}
						\vec{\pi}^\top & = \vec{\pi}^\top  \vec{A}\\
					\end{align*}
				Solving for the stationary distribution is the same as eigenanalysis, where $\vec{\pi}$ is an eigenvector with eigenvalue 1.	
				
				\begin{align*}
					\vec{\pi}^\top(\vec{I} - \vec{A}) = \vec{0}\\
					\vec{\pi}^\top \vec{1} = \vec{1}	
				\end{align*}
				
		\item For a stationary distribution to exist, the markov chain has to be irreducible and aperiodic.
			\begin{itemize}
				\item Irreducibility: The state transition diagram must be singly connected, i.e. it is possible to move from one state to another.
				\item Aperiodicty: $d(i) = gcd\lbrace t: a_{ii}(t)>0\rbrace =1, \forall i$
			\end{itemize}
	
		\item Add detail balance condition	
	
		\item Estimation of the transition matrix from training data (language model)\\
			\begin{enumerate}
				\item The probability of a character is given by
					\begin{align*}
						p(\vec{x}_1\vert \vec{\pi}) & = \prod_{k=1}^K \pi_k^{x_{1k}}\\
						p(\vec{x}_t\vert \vec{x}_{t-1}) & = \prod_{j=1}^K \prod_{k=1}^K a_{jk}^{x_{(t-1)j}x_{tk}}				
					\end{align*}
					
				\item Probability of a sequence $D_l$ with length $T$
					\begin{align*}
						P(D_l\vert \theta) 
						 = p(\vec{x}_1^l, \ldots, \vec{x}_T^l)
						 = p(\vec{x}_1^l) \prod_{t=2}^T p(\vec{x}_t^l \vert \vec{x}_{t-1}^l)
						 = \prod_{k=1}^K \pi_k^{x_{1k}^l}\prod_{t=2}^T\prod_{j=1}^K \prod_{k=1}^K a_{jk}^{x^l_{(t-1)j}x^l_{tk}}	
					\end{align*}
					
				\item The likelihood function is then
					\begin{align*}
						p(D_1,\ldots, D_N\vert \theta) 
						& = \prod_{l=1}^N p(D_l\vert \theta)
						 = \prod_{l=1}^N \left\lbrace  \prod_{k=1}^K \pi_k^{x_{1k}^l}\prod_{t=2}^T\prod_{j=1}^K \prod_{k=1}^K a_{jk}^{x^l_{(t-1)j}x^l_{tk}} \right\rbrace
					\end{align*}
					
					\begin{align*}
						\ln p(D_1,\ldots, D_N\vert \theta) 
						& =\sum_{l=1}^N\sum_{k=1}^K x_{1k}^l\ln\pi_k +\sum_{l=1}^N\sum_{t=2}^T\sum_{j=1}^K \sum_{k=1}^K x^l_{(t-1)j}x^l_{tk}\ln a_{jk}\\
						& = \sum_{k=1}^K \left(\sum_{l=1}^N x_{1k}^l\right)\ln\pi_k +\sum_{j=1}^K \sum_{k=1}^K\left(\sum_{l=1}^N\sum_{t=2}^T x^l_{(t-1)j}x^l_{tk}\right)\ln a_{jk}\\
						& = \sum_{k=1}^K N_k^1 \ln \pi_k + \sum_{j=1}^K \sum_{k=1}^K N_{jk} \ln a_{jk}
					\end{align*}
					
					where 
					\begin{align*}
						N_k^1& = \sum_{l=1}^N x_{1k}^l & N_{jk}&=\sum_{l=1}^N\sum_{t=2}^T x^l_{(t-1)j}x^l_{tk}
					\end{align*}

				\item We just need to solve the optimization below by formulating the Lagrangian:
					\begin{align*}
						\max_{\vec{\theta}}	&\sum_{k=1}^K N_k^1 \ln \pi_k + \sum_{j=1}^K \sum_{k=1}^K N_{jk} \ln a_{jk}\\
						\text{s.t. }&	\sum_{k=1}^K \pi_k =1\\
										&	\sum_{k=1}^K a_{jk} = 1
					\end{align*}									
					
					\begin{align*}
						\pi_k &= \frac{N_k^1}{\sum_{k=1}^K N_k^1}	&	a_{jk} & =\frac{N_{jk}}{\sum_{k=1}^K N_{jk}}
					\end{align*}
				
			\end{enumerate}					
		
			
		
\end{enumerate}




\section{Useful Identities:}

\begin{itemize}
\item Woodbury Identity:
	\begin{align*}
		(\vec{A} + \vec{U}\vec{C}\vec{V})^{-1} = \vec{A}^{-1}-\vec{A}^{-1} \vec{U}(\vec{C}^{-1}+\vec{V}\vec{A}^{-1}\vec{U})\vec{V}\vec{A}^{-1}
	\end{align*}

\item Law of Total Expectation:
	\begin{align*}
		\mathbb{E}_{p(\vec{X})}[\vec{X}] & =\mathbb{E}_{p(\vec{Y})}[\mathbb{E}_{p(\vec{X}\vert\vec{Y})}[\vec{X}\vert \vec{Y}]] 
	\end{align*}


\item Conditional and Margin of Block Distributions
Given the distribution:
\begin{align*}
p(\vec{x},\vec{y}) = \mathcal{N}\left(
\begin{bmatrix}
\vec{\mu}_x\\
\vec{\mu}_y
\end{bmatrix},
\begin{bmatrix}
\vec{\Sigma}_{xx}	& \vec{\Sigma}_{xy}\\
\vec{\Sigma}_{yx}	& \vec{\Sigma}_{yy}
\end{bmatrix}
\right)
\end{align*}

We have
\begin{align*}
p(\vec{x}\vert \vec{y})
& = \mathcal{N}\left(\vec{\mu}_{x\vert y}, \vec{\Sigma}_{x\vert y}\right)\\
\vec{\mu}_{x\vert y}
&=\vec{\mu}_x+ \vec{\Sigma}_{xy}\vec{\Sigma}_{yy}^{-1}(\vec {y} - \vec{\mu}_y)\\
\vec{\Sigma}_{x\vert y}
&=\vec{\Sigma}_{xx} - \vec{\Sigma}_{xy}\vec{\Sigma}_{yy}^{-1}\vec{\Sigma}_{yx}
\end{align*}

\item Product of two Gaussians (note that the underlying random variable must be the same):
\begin{align*}
\mathcal{N}(\vec{x}\vert \vec{a}, \vec{A})\mathcal{N}(\vec{x}\vert \vec{b}, \vec{B})= c\mathcal{N}(\vec{x}\vert \vec{c}, \vec{C})
\end{align*}
\begin{align*}
\vec{C} &= (\vec{A}^{-1}+\vec{B}^{-1})^{-1}\\
\vec{c} &= \vec{C}(\vec{A}^{-1}\vec{a}+\vec{B}^{-1}\vec{b})\\
c& = \mathcal{N}(\vec{a}\vert \vec{b}, \vec{A} + \vec{B}) = \mathcal{N}(\vec{b}\vert \vec{a}, \vec{A} + \vec{B})
\end{align*}

\item Derivative of inverse
\item Derivatve of determinant
\end{itemize}


\end{document}
%%% Local Variables: 
%%% mode: latex
%%% TeX-master: t
%%% End: 
