\documentclass[12pt,twoside]{article}

\newcommand{\reporttitle}{422 Computational Finance}
\newcommand{\reportauthor}{Thomas Teh}
\newcommand{\reporttype}{Notes}
\newcommand{\cid}{0124 3008}

% include files that load packages and define macros
\input{includes} % various packages needed for maths etc.
\input{notation} % short-hand notation and macros


%%%%%%%%%%%%%%%%%%%%%%%%%%%%

\begin{document}
% front page
\input{titlepage}


%%%%%%%%%%%%%%%%%%%%%%%%%%%% Main document
\section{Theory of Interest}

\begin{enumerate}
\item Relationship between effective interest rate $r_{eff}$ and nominal rate ($m$ period compounding), $r^{(m)}$:
	\begin{align*}
		1+r_{eff} = \left(1+\frac{r^{(m)}}{m}\right)^{m}
	\end{align*}

\item \textbf{Definitions.} An ideal bank
	\begin{itemize}
		\item applies the same interest rates for borrowing and lending
		\item no transaction costs
		\item has the same rate for any size of principal
	\end{itemize}

\item \textbf{Definitions.} An ideal bank has an interest value that is independent of the length of time of which it applies, it is called a constant ideal bank.

\item \textbf{Theorem.} The cash flow streams $\lbrace x_i \rbrace_{i=1}^n$ and $\lbrace y_i \rbrace_{i=1}^n$ are equivalent for a constant ideal bank with interest rate $r$ if and only if their PVs are equal. 

\item \textbf{Definitions.} The spot rate $s_t$ is the annualized interest rate charged for money held from the present until time $t$. Properties of spot rate
	\begin{itemize}
		\item Long commitments tend to offer higher interest rates than short commitments
		\item The spot rate curve undulates around in time.
		\item Spot rate curve is normally curved if it is increasing; and inverted if it is decreasing.
		\item Spot rate curve is smooth.
	\end{itemize}

\item \textbf{Definitions.} Forward rate between times $t_1$ and $t_2$ is denoted by $f_{t_1,t_2}$. It is the interest rate charged for borrowing money at time $t_1$ which is to be repaid at $t_2$. $f_{t_1,t_2}$ is agreed on today.
\begin{align*}
	(1+s_{t_2})^{t_2} = (1+s_{t_1})^{t_1}(1+f_{t_1, t_2})^{t_2-t_1}
\end{align*}

\item The forward rate $f_{1,2}$ is 
\begin{itemize}
\item the implied rate for money loaned for 1 year, a year from now
\item the market expectation today of what the 1-year spot rate will be next year
\end{itemize}

\end{enumerate}




\section {Fixed Income Securities}
\begin{enumerate}
	\item Important sequences:
		\begin{itemize}
			\item Geometric Progression
				\begin{align*}
					S_n &= \sum_{k=1}^n ar^{k-1}  =\frac{a(1-r^n)}{1-r}\\
					S_\infty &= \sum_{k=0}^\infty ar^{k-1} = \frac{1}{1-r}\\
				\end{align*}
			\item Arithmetic - Geometric Progression
				\begin{align*}
					S_n & = \sum_{k=1}^n\left[a + (k-1)d\right]r^{k-1}= \frac{a-\left[a+(n-1)d\right]r^n}{1-r}+\frac{dr(1-r^{n-1})}{(1-r)^2}
				\end{align*}
			\end{itemize}
	
	\item Price of various basic securities
		\begin{itemize}
			\item Annuities
				\begin{align*}
					a_{\overline{n}\mid} & = \frac{1-v^n}{r} &
					a_{\overline{n}\mid}^{(m)} & = \frac{1-v^n}{r^{m}}
				\end{align*}
			\item Perpetuity
				\begin{align*}
					a_{\overline{\infty}\mid} & = \frac{1}{r}&
					a_{\overline{\infty}\mid} & = \frac{1}{r^{m}}\\
				\end{align*}
				
			\item  Varying annuity
				\begin{align*}
					(Ia)_{\overline{n}\mid} & = Pa_{\overline{n}\mid} + D\left[\frac{a_{\overline{n}\mid} - nv^n}{i}\right]
				\end{align*}
				
			\item Bond
				\begin{align*}
					P & = NCa_{\overline{n}\mid} + Nv^n
				\end{align*}
		\end{itemize}

	\item \textbf{Definition.} A bond's \textbf{yield to maturity} if the flat interest rate at which the PV of the CFs is equal to the current price. The bond's \textbf{current yield} is $\frac{NC}{P}$.
		\begin{itemize}
			\item There's an inverse relationship between price and yield
			\item The longer the time to maturity, the more sensitive is the price of the bond to the yield (think of duration!)
			\item If yield to maturity is the same as the coupon rate, the bond price would be the face value
		\end{itemize}
		
	\item \textbf{Duration.} Duration measures the sensitivity of the bond with respect to the interest rate
		\begin{align*}
			D_{mac}& = \frac{\sum_{t=1}^n t\times PV_t }{\sum_{t=1}^n PV_t}\\
			D_{mod}& = -\frac{1}{P(r_0)}\left.\frac{dP(r)}{dr}\right\vert_{r=r_0} = \frac{D_{mac}}{(1+r_0)}
		\end{align*}
		
	\item \textbf{Convexity.} Convexity is defined as
		\begin{align*}
			C	&=\frac{1}{P(r_0)}\left.\frac{d^2P(r)}{dr^2}\right\vert_{r=r_0}
				=\frac{1}{P(r_0)}\sum_{t=1}^n t^2PV_t
		\end{align*}
	
	\item Estimation of bond price change
		\begin{align*}
			\Delta P \approx -D_{mod}P(r_0)\Delta r + \frac{1}{2}P(r_0) (\Delta r)^2
		\end{align*}

	\item \textbf{Immunization.} Let $A$ be assets and $L$ be liabilities.
		\begin{itemize}
			\item $PV(A) = PV(L)$
			\item $D(A) = D(L)$
			\item $C(A) > C(L)$
		\end{itemize}

\end{enumerate}




Macaulay Duration


Modified Duration




\end{document}
%%% Local Variables: 
%%% mode: latex
%%% TeX-master: t
%%% End: 
