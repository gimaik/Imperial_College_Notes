\documentclass[12pt,twoside]{article}

\newcommand{\reporttitle}{422 Computational Finance}
\newcommand{\reportauthor}{Thomas Teh}
\newcommand{\reporttype}{Notes}
\newcommand{\cid}{0124 3008}

% include files that load packages and define macros
\input{includes} % various packages needed for maths etc.
\input{notation} % short-hand notation and macros


%%%%%%%%%%%%%%%%%%%%%%%%%%%%

\begin{document}
% front page
\input{titlepage}


%%%%%%%%%%%%%%%%%%%%%%%%%%%% Main document
\section {Useful Identities}

Geometric Progression
\begin{align*}
S_n &= \sum_{k=1}^n ar^{k-1}  =\frac{a(1-r^n)}{1-r}\\
S_\infty &= \sum_{k=0}^\infty ar^{k-1} = \frac{1}{1-r}\\
\end{align*}

Arithmetic - Geometric Progression
\begin{align*}
S_n & = \sum_{k=1}^n\left[a + (k-1)d\right]r^{k-1}= \frac{a-\left[a+(n-1)d\right]r^n}{1-r}+\frac{dr(1-r^{n-1})}{(1-r)^2}
\end{align*}

Price of an Annuity
\begin{align*}
a_{\overline{n}\mid} & = \frac{1-v^n}{i}\\
a_{\overline{n}\mid}^{(m)} & = \frac{1-v^n}{r^{m}}\\
\end{align*}

Price of a Perpetuity
\begin{align*}
a_{\overline{\infty}\mid} & = \frac{1}{i}\\
a_{\overline{\infty}\mid} & = \frac{1}{r^{m}}\\
\end{align*}

Price of a varying annuity
\begin{align*}
(Ia)_{\overline{n}\mid} & = Pa_{\overline{n}\mid} + D\left[\frac{a_{\overline{n}\mid} - nv^n}{i}\right]
\end{align*}

Price of a bond
\begin{align*}
P & = NCa_{\overline{n}\mid} + Nv^n
\end{align*}

Macaulay Duration


Modified Duration




\end{document}
%%% Local Variables: 
%%% mode: latex
%%% TeX-master: t
%%% End: 
